%! TeX program = lualatex

\documentclass{beamer}

\beamerdefaultoverlayspecification{<+->}

\usepackage{amsmath}
\usepackage{fontspec}
\usepackage{fancyvrb}

\begin{document}

\setmonofont{Jet Brains Mono}[Scale=MatchAveragecase]

% \newcommand{\underverb}[2] {$\underbrace{\verb+#1+}_{\textrm{#2}}$}
\newcommand{\blue}[2] {\textcolor{blue}{#1}}
\newcommand{\green}[2] {\textcolor{green}{#1}}
\newcommand{\intermezzo}[1]{\title{#1}\author{}\date{}\begin{frame}\titlepage\end{frame}}

\title{Jacinda---Implementing an Efficient Functional Stream Processing Language}
\author{Vanessa McHale}
\date{11 Oct 2024}

\begin{frame}
\titlepage
\end{frame}

\begin{frame}
  \frametitle{Outline}
  \begin{enumerate}
    \item Unix and Functional Programming
      \begin{itemize}
        \item Regular Expressions
      % Talk up awk? really popular!
      \end{itemize}
    \item Show Off
    \item Implementation
      \begin{itemize}
        \item Haskell
      \end{itemize}
  \end{enumerate}
\end{frame}

\begin{frame}[fragile]
  \frametitle{Unix Command-Line}
  \begin{itemize}
    \item
      \begin{verbatim}
~ % ps
  PID   TT  STAT      TIME COMMAND
  ⋮
22970 s002  S      0:00.07 -zsh
22989 s002  S+     0:00.21 cabal repl lib:apple
23031 s002  S+     0:00.03 /Users/vanessa/.ghcup/bin/cabal
23033 s002  S+     0:10.46 /Users/vanessa/.ghcup/ghc/9.10.1/lib/
      \end{verbatim}
      \item
        \begin{verbatim}
~ % ps | rg cabal | rg -v rg
23400 s002  S+     0:00.16 cabal repl lib:apple
23401 s002  S+     0:00.04 /Users/vanessa/.ghcup/bin/cabal
23403 s002  S+     0:11.17 /Users/vanessa/.ghcup/ghc/9.10.1/lib
        \end{verbatim}
  \end{itemize}
\end{frame}

\begin{frame}[fragile]
  \frametitle{Unix Command-Line}
  \begin{itemize}
      \item
        \begin{verbatim}
~ % ps | rg 'cabal' | rg -v 'rg' | cut -d' ' -f1
24144
24196
24198
        \end{verbatim}
      \item
        \begin{verbatim}
~ % ps | rg cabal | rg -v rg | \
      cut -d' ' -f1 | xargs kill
        \end{verbatim}
    \end{itemize}
\end{frame}

\begin{frame}[fragile]
  \frametitle{Stuctured Text in Unix}
    \begin{verbatim}
 % otool -l $(locate librure.dylib)
⋮
Load command 12
          cmd LC_LOAD_DYLIB
      cmdsize 56
         name /usr/lib/libiconv.2.dylib (offset 24)
   time stamp 2 Wed Dec 31 19:00:02 1969
      current version 7.0.0
compatibility version 7.0.0
Load command 13
          cmd LC_LOAD_DYLIB
      cmdsize 56
         name /usr/lib/libSystem.B.dylib (offset 24)
   time stamp 2 Wed Dec 31 19:00:02 1969
      current version 1351.0.0
⋮
    \end{verbatim}

\end{frame}

\begin{frame}[fragile]
  \frametitle{Structured Text in Unix}
    \begin{Verbatim}[commandchars=\\\{\}]
 % otool -l $(locate librure.dylib)
⋮
Load command 12
          cmd LC_LOAD_DYLIB
      cmdsize 56
         \blue{name}  /usr/lib/libiconv.2.dylib (offset 24)
   time stamp 2 Wed Dec 31 19:00:02 1969
      current version 7.0.0
compatibility version 7.0.0
Load command 13
          cmd LC_LOAD_DYLIB
      cmdsize 56
         \blue{name}  /usr/lib/libSystem.B.dylib (offset 24)
   time stamp 2 Wed Dec 31 19:00:02 1969
      current version 1351.0.0
⋮
    \end{Verbatim}
\end{frame}

\begin{frame}[fragile]
  \frametitle{Structured Text in Unix}
    \begin{Verbatim}[commandchars=\\\{\}]
 % otool -l $(locate librure.dylib)
⋮
Load command 12
          cmd LC_LOAD_DYLIB
      cmdsize 56
         \blue{name}  \green{/usr/lib/libiconv.2.dylib}  (offset 24)
   time stamp 2 Wed Dec 31 19:00:02 1969
      current version 7.0.0
compatibility version 7.0.0
Load command 13
          cmd LC_LOAD_DYLIB
      cmdsize 56
         \blue{name}  \green{/usr/lib/libSystem.B.dylib}  (offset 24)
   time stamp 2 Wed Dec 31 19:00:02 1969
      current version 1351.0.0
⋮
    \end{Verbatim}
\end{frame}

% printenv | ja -F= '{% /^PATH/}{`2}'

\begin{frame}[fragile]
  \frametitle{Structured Text---Patterns}
  \begin{itemize}
    \item
      \begin{verbatim}
~ % otool -l $(locate librure.dylib) | \
      awk '$1 ~ /^name/ {print $2}'
      \end{verbatim}
      % halign?
    \item \vtop{\ialign {#\crcr \hfil \verb|$1 ~ /^name/| \hfil \crcr \noalign {\kern 3\dimen11 \nointerlineskip} \upbracefill \crcr \noalign {\kern 3\dimen11}}}{\textsubscript{Execute on these lines}}
    % \item AWK: split into records (lines), split by fields
  \end{itemize}
\end{frame}

\begin{frame}[fragile]
  \frametitle{Structured Text---Jacinda}
  \begin{verbatim}
~ % otool -l $(locate librure.dylib) | \
      ja '{`1 ~ /^name/}{`2}'
/usr/local/lib/librure.dylib
/usr/lib/libiconv.2.dylib
/usr/lib/libSystem.B.dylib
  \end{verbatim}
\end{frame}

\intermezzo{Implementation}

\begin{frame}[fragile]
  \frametitle{Crash Course}
  \framesubtitle{Polymorphic Syntax Trees}
    \begin{itemize}
      \item
      \begin{verbatim}
data Expr a = RealLit a !Double
            | Var a !Name
            | Lam a Name (Expr a)
            ⋮
\end{verbatim}
      \item \verb|parse :: String -> Expr Loc|
      \item Functorial!
        \begin{itemize}
          \item \verb|void :: Expr Loc -> Expr ()|
          \item \verb|DeriveFunctor|
        \end{itemize}
    \end{itemize}
\end{frame}

\begin{frame}[fragile]
  \frametitle{Crash Course}
  \framesubtitle{No Symbol Table}
  \begin{itemize}
      \item Annotate AST with types
        \begin{verbatim}
tyOf :: Expr Loc -> Either TyErr (Expr Type)
        \end{verbatim}
      \item
        \begin{verbatim}
applySubstE :: Subst -> Expr Type -> Expr Type
applySubstE s = fmap (applySubst s)
        \end{verbatim}
      \item GHC approach
  \end{itemize}
\end{frame}

% objdump -d /usr/bin/* | cut -f3 | ja '~.{%/^[a-z]+/}{`1}'
% pbpaste | ja '.?{|`1 ~* 1 /([^\?]*)/}'
% https://github.com/vmchale/jacinda/blob/canon/EXAMPLES.md#extract-library-versions-unstripped

% Chess example?

\end{document}
